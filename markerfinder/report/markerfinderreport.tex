\documentclass{article}

\usepackage{amsmath}
\usepackage{amsthm}
\usepackage{graphicx}
\usepackage{hyperref}
\setlength{\parskip}{1em}

\begin{document}

	\title{ECOAR project report\\Visual marker detection utility
	\\written in MIPS assembly}
	\author{Michał Szopiński}
	\date{December 18, 2020}
	\maketitle
	
	\setcounter{section}{-1}
	\section{Abstract}
	
	This document discusses the implementation details of a MIPS program to
	detect and locate a specific pattern of pixels within a bitmap.
	
	The implementation utilizes a breadth-first search algorithm to quickly
	discern potential candidates and applies a separate validation subroutine to
	decide whether a candidate matches the characteristics of a marker.
	
	In addition, the program implements a simple BMP/DIB file parser with basic
	error handling.
	
	\newpage
	\renewcommand{\baselinestretch}{0}\normalsize
	\tableofcontents
	\renewcommand{\baselinestretch}{1}\normalsize
	
	\newpage
	\section{Project objectives}
	
	The input of the program is an uncompressed 24-bit BMP/DIB file named
	\texttt{source.bmp}. The image dimensions are $320\times240$ px at the most.
	
	Contained within the bitmap are \textbf{markers}, which are groups of pixels
	that match a specific set of characteristics. An image may contain up to 50
	such markers, although this implementation has no limitations on the number
	of markers it can detect.
	
	Examples of valid markers:
	
	\begin{center}
		\fbox{\includegraphics[width=0.5\textwidth]{markers/ex1}}
	\end{center}
	
	The location of a marker is given by the point of intersection of its arms,
	i.e. the top right corner of the figure. The coordinates of each marker
	are to be printed on separate lines, each coordinate separated by a comma
	and a space.
	
	\subsection{Constraints of a marker}
	
	Because the original task is vague about what precisely constitutes a
	marker, a semi-formal definition of a marker must be introduced to be
	followed by this implementation.
	
	\newtheorem{definition}{Definition}
	\begin{definition}
		\label{def:marker}
  		A marker is a continuous group of black pixels that can be constructed
		by drawing a black square and cutting away a smaller square from its
		bottom left corner. The group must not directly neighbor any black
		pixel.
	\end{definition}
	
	\newpage
	\subsection{Examples of valid markers}
	
	What follows from the above definition is that markers may vary in size
	(width and height) and thickness:
	\begin{figure}[h]
		\centering
		\fbox{\includegraphics[width=0.5\textwidth]{markers/exsize}}
		\caption{Valid markers of varying size and constant thickness}
	\end{figure}
	\begin{figure}[h]
		\centering
		\fbox{\includegraphics[width=0.5\textwidth]{markers/exthick}}
		\caption{Valid markers of varying thickness and constant size}
	\end{figure}
	
	Furthermore, the definition places no constraints on the white space
	surrounding the markers (short of direct adjacency), which means that the
	rectangle occupied by a marker may also contain other black pixels.
	
	\begin{figure}[h]
		\centering
		\fbox{\includegraphics[width=0.5\textwidth]{markers/exdistractor}}
		\caption{Two valid markers with ``distractors"}
		\label{fig:distractors}
	\end{figure}
	
	\newpage
	Markers may also overlap.
	
	\begin{figure}[h]
		\centering
		\fbox{\includegraphics[width=0.5\textwidth]{markers/exoverlap}}
		\caption{Three overlapping markers}
	\end{figure}
	
	\subsection{Examples of invalid markers}
	
	Markers must not touch other black pixels (diagonal neighbors are
	acceptable).
	
	\begin{figure}[h]
		\centering
		\fbox{\includegraphics[width=0.5\textwidth]{markers/extouch}}
		\caption{Markers invalidated by direct neighbors}
	\end{figure}
	
	Markers must be continuous and obtainable by cutting away a square from a
	square.
	
	\begin{figure}[h]
		\centering
		\fbox{\includegraphics[width=0.5\textwidth]{markers/exirreg}}
		\caption{Markers invalidated by irregularities}
	\end{figure}
	
	\newpage
	\section{Program flow overview}
	
	\subsection{Modules}
	
	The program is modular in nature, with each module providing distinct data
	processing functionality. The main module is the entry point of the program.
	It coordinates the exchange of data between each of its child modules.
	
	\begin{figure}[h]
		\centering
		\includegraphics[width=0.75\textwidth]{figs/structure}
		\caption{Module hierarchy}
	\end{figure}
	
	Each module exposes a global subroutine for the parent module to call as
	necessary.
	
	\subsection{Top-level algorithm}
	\label{sec:topalg}
	
	The main module executes the following simple algorithm to print the
	coordinates of all markers within the bitmap:
	
	\begin{enumerate}
		\item The main module statically allocates the image buffer, the chunk
		boundary buffer and certain text constants, including the input
		file name.
		\item The BMP reader is invoked to load the bitmap into the buffer. If a
		read error occurs, the program is terminated.
		\item The chunk finder attempts to find an unvisited \textit{chunk},
		i.e. a continuous group of black pixels. If there are none, the program
		is terminated.
		\item The found chunk is validated by the validator module. If it
		matches the characteristics of a marker, its coordinates are printed.
		\item Step 3 is repeated.
	\end{enumerate}
	
	\section{BMP reader}
	
	\subsection{File IO and header validation}
	
	A handle to the input file is obtained by performing a \texttt{syscall} with
	the argument $13$. If the returned handle is negative, the operation has
	failed and the program terminates with an error message.
	
	The bitmap metadata is extracted based on the following offsets:
	\begin{center}
		\begin{tabular}{|c|l|}
			\hline
			Offset & Description \\
			\hline
			0 & BMP format marker\\
			14 & DIB header size \\
			18 & Image width \\
			22 & Image height \\
			28 & Bits per pixel \\
			30 & Compression algorithm \\
			54 & Raster data \\
			\hline
		\end{tabular}
	\end{center}
	The header is validated according to the following rules:
	\begin{enumerate}
		\item If the format marker doesn't match the string \texttt{BM}
		(little-endian value \texttt{0x4D42}), terminate with an ``unrecognized
		file format" error.
		\item If the header size isn't 40 (standard Windows NT format) or the
		compression isn't 0 (none), terminate with an ``unsupported BMP format"
		error.
		\item If the bits per pixel number isn't 24, terminate with an
		``unsupported bit depth" error.
	\end{enumerate}
	
	\subsection{Raster data extraction}
	
	Because a BMP file may contain additional metadata at the end, the number
	of pixels to be read is calculated from the width and the height of the
	image.
	
	The reading process starts by obtaining a pointer to the buffer element
	corresponding to the lower left corner of the bitmap. Pixels are then read
	in lines, and with each pixel, the pointer moves to the right.
	
	The RGB composition of each pixel is analyzed and, if it is all-black, the
	value 1 is assigned to its position in the buffer. A non-black pixel has an
	assigned value of 0.
	
	\begin{figure}[h]
		\centering
		\fbox{\includegraphics[width=0.5\textwidth]{figs/buffer}}
		\caption{A visual representation of how a bitmap is loaded into the
		buffer. Pixel marked 1 is loaded first, pixel marked 2 is loaded last.
		Pixel 0 marks the start of the buffer.}
	\end{figure}
	
	Once the end of a line is reached, the buffer pointer is moved to the start
	of the preceding line. Moreover, the byte padding present at the end of
	rows in an uncompressed BMP file is read and discarded.
	
	Finally, once all pixels have been read, the file handle is closed and
	control is returned to the main module.
	
	\newpage
	\section{Chunk finder}
	
	\subsection{Chunk discovery (buffer scanning)}
	
	As described in section~\ref{sec:topalg}, the program repeatedly searches
	for unvisited candidates for markers.
	
	The algorithm for finding them is very simple: the bitmap is searched from
	the top left to the bottom right, and each time a value of 1 is discovered,
	the index of discovery is noted for further processing.
	
	If the end of the buffer is reached before a chunk is found, the search
	stops and the subroutine returns a value of 0 to the caller.
	
	To optimize the operation, the chunk finding module keeps track of the next
	index to start the search from. This value is stored in a statically
	allocated variable.
	
	\begin{figure}[h]
		\centering
		\fbox{\includegraphics[width=0.5\textwidth]{figs/scan}}
		\caption{Illustration marking the order and location of discovery}
	\end{figure}
	
	The reason why this approach doesn't result in the repeated discovery of the
	same chunk is detailed in the next section.
	
	\newpage
	\subsection{Chunk exploration (breadth-first search)}
	\label{sec:bfs}
	
	Immediately following the discovery of a chunk, an exploration subroutine
	is invoked to determine the shape and the bounding rectangle of the chunk.
	
	Knowing a single initial point of entry, the remaining pixels of a chunk can
	be discovered by implementing a breadth-first search (BFS) algorithm on the
	two-dimensional graph represented by the buffer.
	
	The algorithm inspects the immediate neighbors of the entry point and checks
	their color (their value in the buffer) to determine whether they are a part
	of the newly discovered chunk. If so, their subsequent neighbors are also
	queued for inspection and the algorithm reiterates.
	
	\begin{figure}[h]
		\centering
		\fbox{\includegraphics[width=0.5\textwidth]{figs/bfs}}
		\caption{The order of inspection using BFS. Lighter pixels mark elements
		inspected later. Zeros mark entry points. Ones mark elements currently
		queued for inspection.}
	\end{figure}
	
	Upon inspection, all discovered elements, including the entry point, are
	assigned a color of 2. This serves several purposes:
	\begin{enumerate}
		\item Prevents the BFS from infinitely looping over the same nodes,
		since only elements with a color of 1 are queued for inspection.
		\item Allows the validation algorithm to discern local black pixels
		from foreign black pixels (``distractors" demonstrated in
		fig.~\ref{fig:distractors}).
		\item Prevents the chunk discovery algorithm from discovering the same
		chunk more than once.
	\end{enumerate}
	
	The BFS algorithm operates on the framework of coordinates, which can be
	translated to buffer indices when the time comes to inspect the color of an
	element. This allows the program to update the bounding rectangle of a chunk
	whenever a node is visited.
	
	Once the inspection queue is empty, the algorithm terminates. The bounding
	rectangle and the updated image buffer can then be passed to the validation
	subroutine for pattern matching.
	
	\subsection{FIFO queue}
	
	Frequently mentioned in section~\ref{sec:bfs} is the concept of ``queuing"
	elements for inspection. In this context, ``queuing" means storing them and
	retrieving them in the order in which they were stored, i.e. first in, first
	out (FIFO).
	
	The data structure providing this functionality is implemented as a
	continuous memory region combined with two indices: a start index and an
	end index. The basic queue operations are implemented as follows:
	\begin{itemize}
		\item Push -- The new element is placed at the end index and the end
		index is incremented.
		\item Pop -- The element is retrieved from the start index and the start
		index is incremented.
		\item IsEmpty -- The queue is considered empty if the start index is
		equal to the end index.
	\end{itemize}
	
	If, during incrementation, one of the indices exceeds the bounds of the
	memory block, the offending index is reset to zero. This allows the queue to
	wrap around its physical limits and store as many elements as can fit inside
	the region.
	
	\begin{figure}[h]
		\centering
		\fbox{\includegraphics[width=0.5\textwidth]{figs/queue}}
		\caption{A visual representation of a queue wrapping around its bounds.
		0 marks the start index, 1 marks the end index, dark gray marks the
		elements, light gray marks free space.}
	\end{figure}
	
	Because the queue is used to store pairs of coordinates, the module
	implementing the queue accepts arguments and returns values in pairs of
	half-words. The queue itself is defined as an array of words, and the proper
	address translation is done at access time.
	
	\newpage
	\section{Validator}
	
	The validation subroutine inspects the updated image buffer and the chunk's
	bounding rectangle to determine whether the selected chunk constitutes a
	valid marker.
	
	The output of this subroutine is a Boolean value denoting the validity of
	the marker. Furthermore, the image buffer is updated and all elements of
	color 2 are assigned a color of 3. Again, this serves to differentiate
	between local and foreign black pixels during later calls to the subroutine.
	
	In the context of the validity check, ``black" refers to the color 2.
	Any pixel with a color other than 2, including 1 and 3, is considered white.
	
	\subsection{Initial checks}
	
	Two simple checks can be conducted to immediately reject candidates before
	thoroughly inspecting their body. If any of the following is true, the
	candidate is rejected:
	\begin{enumerate}
		\item The bounding rectangle isn't square.
		\item The top right corner isn't black.
	\end{enumerate}
	These conditions follow from def.~\ref{def:marker}.
	
	\subsection{Radial continuity check}
	
	Having established that the candidate may have two arms of equal length and
	a thickness of at least one, all that remains is to check its body for
	discontinuities, i.e. white pixels inside the arms or black pixels outside
	the arms.
	
	The check starts at the top right corner and verifies the continuity of the
	horizontal arm, followed by the vertical arm. After that, it moves
	diagonally to the next intersection and repeats. The offset of the start
	point from the top right corner is referred to as the ``radius".
	
	\newpage
	\begin{figure}[h]
		\centering
		\fbox{\includegraphics[width=0.5\textwidth]{figs/radius}}
		\caption{A schematic of the continuity check. Light gray marks the
		horizontal arm, dark gray marks the vertical arm, numbers mark the
		radius.}
	\end{figure}
	
	At the start of each radius, the color of the intersection is probed. If the
	color is white, it means that the continuity of the arms has been verified
	and from now on, only white pixels are allowed.
	
	Once the vertical arm of the final radius has been verified, one final check
	is performed. If the radial check hasn't detected a white intersection by
	now, what follows is that the chunk is solid black. As such, it fails to
	meet the criteria of def.~\ref{def:marker} and is rejected as a marker.
	
	If all the checks have been passed, control is returned to the main module
	and the coordinates of the marker are printed.
	
\end{document}
